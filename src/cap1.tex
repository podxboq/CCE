\section{Contexto y estado del arte}

Cuando un mensaje es enviado, aún cuando se pueda garantizar su recepción, se debe también asegurar que el mensaje no ha sido alterado o que si ha sido modificado, se tiene que poder recuperar la información original.

Para usar un canal de transmisión digital, el mensaje a de ser codificado usando un conjunto finito de símbolos que recibe el nombre de \textbf{alfabeto}, a lo largo de este trabajo, se denota a este conjunto con la letra $\Gamma$ y por $q$ a su longitud o cardinal, $q=|\Gamma|$.
Se prestará especial atención al alfabeto digital por excelencia, $\Gamma = \Z_q = \set{0, 1}$ también llamado alfabeto binario, pues es el usado en todas las transmisiones a través de internet.

Hay que recordar, que en general sobre $\Z_q$ se tiene definido una suma interna, muchas veces llamada suma directa y que se denota como $\oplus$ y que está definido como $x\oplus y = (x+y)\bmod q$.

Un resultado trivial y que se debe resaltar es que $\forall\ x,y\in\Z_q$ o bien $x=y$ o bien $x\oplus 1=y$.

El mensaje codificado es enviado a través de un canal, y en el mejor de los casos llega al receptor sin alteración del contenido.
Sin embargo la situación real es que no podemos garantizar la no modificación del mensaje, a esta situación se conoce como un \textbf{canal con ruido}.

Así se hace invitable que el receptor tenga algún mecanismo que permita decodificar el mensaje recibido y recuperar el mensaje original, estas técnicas son aplicadas por el emisor con el fin de proteger su codificación ante un número de alteraciones lo suficientemente adecuado al ruido que tenga el canal a usar.

Este trabajo se centrará en una de estas técnicas conocidas como \textbf{códigos de corrección de errores} que consiste en asignar a cada elemento del alfabeto, un bloque de una longitud fijada de elementos del mismo alfabeto de tal forma que pequeños cambios en el bloque siga permitiendo identificar el bloque original y por tanto el símbolo que se quería enviar.

\begin{definition}
	Sea $\Gamma$ un alfabeto de longitud $q$ y $n\in\N$, un \textbf{$q$-código de longitud $n$} es cualquier subconjunto de $\Gamma^n$.
	Los $2$-códigos también se llaman \textbf{códigos binarios}.
\end{definition}

\begin{definition}
	Sea $\Gamma$ un alfabeto de longitud $q$ y $C$ un $q$-código de longitud $n$.
	Una \textbf{palabra} de longitud $n$ o simplemente palabra, es un elemento de $\Gamma^n$, que se llama \textbf{palabra clave} si es un elemento de $C$.
\end{definition}

Una palabra $(x_1,\dots,x_n)\in\Gamma^n$ para simplificar notación se podrá escribir como $x_1\cdots x_n$, y cuando todos los elementos $x_i$ sean iguales a $x$ también se podrá escribir por $\logical{x}{n}$.

La idea detrás de la técnica de los códigos de corrección de errores es tener asociado a un alfabeto $\Gamma$ de longitud $q$ un $q$-código $C$ tal que podamos asociar a cada elemento del alfabeto un elemento del código y que los cambios en la palabra clave sigan identificando el elemento del alfabeto.

\begin{example}
	Sea $\Gamma=\set{1, 2, 3, 4, 5, 6}$ los valores de lanzar un dado, y se define el $6$-código de longitud 5, $C=\set{11111, 22222, 33333, 44444, 55555, 66666}$.
	Tras lanzar el dado 3 veces y obtener el resultado 1, 2, 5 se codifica como 111112222233333 y es enviado por un canal con ruido que modifica el mensaje y es recibido como 114142235233433.

	Con esta técnica podemos recuperar el mensaje original pues es fácil ver como era el mensaje original
	\begin{align*}
		11414 & \so 11111 \\
		22352 & \so 22222 \\
		33433 & \so 33333
	\end{align*}
\end{example}

Se denota por $\push{x}$ a una palabra recibida a través de un canal con ruido cuando se ha enviado una palabra clave $x\in C$.
Análogamente se denota por $\pull{x}$ a la palabra clave usada en el envío de una palabra recibida $x$.

Así que podemos resumir que el objetivo de los códigos de corrección de errores es identificar bajo que condiciones dado $\Gamma$ un alfabeto de longitud $q$ se puede construir $C$ un $q$-código de longitud $n$ para el cual
\[
	\forall y\in\Gamma^n\ \exists!\ x\in C \tq y = \push{x}
\]

\subsection{Distancia de Hamming}

Para cualquier conjunto finito $\Gamma$ de $q$ elementos, se puede identificar $\Gamma$ unívocamente con $\Z_q$, así que a partir de este momento siempre usaremos $\Z_q$ como alfabeto de trabajo de longitud $q$.

La manera más sencilla de proteger y recuperar la información que se quiere transmitir, es enviando varias copias del elemento inicial, y comparar todas las copias entre sí, la cantidad suficiente de veces que nos permita recuperar el valor modificado en caso de error.

Para cualquier elemento $x\in\Z_q$ del alfabeto, se codifica mediante repetición en un bloque de longitud $n$, nuestro $q$-código $C$ será el conjunto $\set{\logical{x}{n}\tq x\in\Z_q}\subset \Z_q^n$, este tipo de código recibe el nombre de \textbf{$q$-código de repetición de longitud $n$}, de esta manera, si la recepción de la palabra $x$ es tal que no pertenece a $C$, se sabe que ha ocurrido un error, y en algunos casos, se sabrá corregir el error y obtener el elemento originalmente enviado.

Otro detalle importante, que se debe tener en cuenta es si $n$ es par y $m=n/2$, el elemento recibido $\logical{x}{m}\logical{y}{m}$ no se puede corregir, ya que el mismo número de errores se han producido tanto si el elemento original es $\logical{x}{n}$ como $\logical{y}{n}$.

Tras las reflexiones anteriores, se van a definir los conceptos que usaremos a lo largo de este trabajo para abordar el objetivo.

\begin{definition}
	Sean $x=\palabraG$ y $y=\palabraN{y}$ dos palabras de $\Z_q^n$, se llama \define{distancia de Hamming}{distancia_hamming} a
	\[
		d(x, y) = x_1\oplus y_1 + \dots + x_n\oplus y_n
	\]
\end{definition}

Como se comprueba en Hill~(\cite{hill_first_1980}) la distancia de Hamming es una distancia, y como es la única distancia que se va a considerar se llamará simplemente distancia.

Cuando un elemento $x\in\Z_q^n$ es enviado y el canal lo modifica, lo importante no es cuantos cambios se realizan en $x$, pues a veces un cambio corrige un cambio anterior, así que simplemente hay que saber como está el elemento en el momento de recibirlo, y la mejor forma de medir estos cambios es sabiendo la distancia con respecto al enviado.

\begin{definition}
	Sean $C$ un código de $\Z_q^n$ y $x\in C$, se dice que $\push{x}$ es un \define{n-error}{n_error} si $d(x, \push{x})= n$.
\end{definition}

Se puede definir también la distancia de una palabra a un conjunto y dos valores asociado a un conjunto que se llama el radio interior y el radio exterior.

\begin{definition}
	Sean $x=\palabraG$ y $C$ un subconjunto de $\Z_q^n$, se llama \define{distancia de x a C}{distancia_conjunto} a
	\[
		d(x, C) = \min\set{d(x,y)\ \forall y\in C}
	\]
\end{definition}

\begin{definition}
	Sean $C$ un subconjunto de $\Z_q^n$ con al menos dos elementos, se llama \define{radio interior de C}{radio_interior_conjunto} a
	\[
		r(C) = \min\set{d(x,y)\ \forall x,y\in C\tq x\neq y}
	\]
\end{definition}

\begin{definition}
	Sean $C$ un subconjunto de $\Z_q^n$, se llama \define{radio exterior de C}{radio_exterior_conjunto} a
	\[
		R(C) = \max\set{d(x,y)\ \forall x,y\in C}
	\]
\end{definition}

\begin{definition}
	Sean $x\in\Z_q^n$, se llama \define{bola de centro x y radio n}{bola} a
	\[
		B(x, n) = \set{y\in\Z_q^n \tq d(x,y)\leq n}
	\]
\end{definition}

Con la nomenclatura de las definiciones anteriores, se puede comprobar fácilmente los siguientes resultados:
\begin{enumerate}
	\item $x\in C \iff d(x, C) = 0$.
	\item $r(C)\leq R(C) \leq n$.
\end{enumerate}

Una técnica muy habitual de corregir errores es considerar que el elemento recibido, originalmente es el elemento más cercano al recibido, es decir, si $C$ es un código y $x\in C$ es enviado, entonces tomamos $n=d(\push{x}, C)$ y un elemento $x'\in C\tq d(\push{x}, x')=n$, hay que considerar códigos para los cuales sin opción se tenga que $x'=x$.

\begin{lemma}
	Sea $C\in\Z_q^2$ un código con radio interno $n$, si $n$ es impar, entonces se puede corregir correctamente a lo más los $\ent{n/2}$-errores.
\end{lemma}
\begin{proof}
	Sea $x\in C$ para el cual $\pull{x}$ es un $\ent{n/2}$-error, entonces $d(x, \push{x})=\ent{n/2}$.
	Supongamos que existe otra palabra $y$ del código que es igual o más cercano a $\push{x}$, entonces
	\[
		n = r(C) \leq d(x, y) \leq d(x, \push{x}) + d(\push{x}, y) = \ent{n/2} + d(\push{x}, y) \leq \ent{n/2} + \ent{n/2} < n
	\]
	Por tanto se llega a una contradicción, así que
	\[
		C\cap B(\push{x}, \ent{n/2}) = \set{x}\qedhere
	\]
\end{proof}

Con el mismo razonamiento, se puede observar que para cuando el radio interno de un código es $n$ par, entonces se puede corregir a lo más $\ent{(n-1)/2}$-errores.
\begin{example}
	\label{ex:1}
	Consideremos el siguiente código de $\Z_2^4$, $C=\set{0000, 0101, 1010, 1111}$.
	\begin{align*}
		d(0000, 0101) &= 2 & d(0000, 1010) &= 2 & d(0000, 1111) = 4 \\
		& & d(0101, 1010) &= 2 & d(0101, 1111) = 2 \\
		& & & & d(1010, 1111) = 2
	\end{align*}
	Tenemos que el radio interno de $C$ es $2$, como $\ent{(2-1)/2}=0$ pues es claro que hay errores que no se pueden corregir correctamente.

	Efectivamente, si por ejemplo recibimos la palabra $1011$, no se puede saber si la palabra clave enviada es $1010$ o $1111$, es decir, puede haber ocurrido un sólo error en la cuarta posición o un sólo error en la segunda posición.
\end{example}

\begin{example}
	\label{ex:2}
	Consideremos el siguiente código de $\Z_2^5$, $C=\set{00000, 10101, 11010, 01111}$.
	\begin{align*}
		d(00000, 10101) &= 3 & d(00000, 11010) &= 3 & d(00000, 01111) = 4 \\
		& & d(10101, 11010) &= 3 & d(10101, 01111) = 3 \\
		& & & & d(11010, 01111) = 3
	\end{align*}
	Tenemos que el radio interno de $C$ es $3$, como $\ent{3/2}=1$ pues se pueden corregir correctamente cualquier 1-error.

	Por ejemplo, si recibimos la palabra $10111$, calculando la distancia a $C$ que es
	\begin{align*}
		d(10111, 00000) &= 4 & d(10111, 10101) &= 1 & d(10111, 11010) &= 3 & d(10111, 01111) = 2
	\end{align*}
	Así que el menor de las distancias es únicamente $10101$ y por tanto podemos corregir el error.
\end{example}

\begin{definition}
	Diremos que $C$ es un $(n, m, d)$-código si $C\subset\Z_q^n$, $|C| = m$ y $r(C)=d$.
\end{definition}
Así el ejemplo~\ref{ex:1} es un $(4, 4, 2)$-código, mientras el ejemplo~\ref{ex:2} es un $(5, 4, 3)$-código.
Los códigos binarios de repetición de longitud n son $(n, 2, n)$-códigos.

Hasta el momento se ha considerado que los errores en la trasmisión se realizan de forma aleatoria y equiprobable sobre cualquier símbolo, este tipo de canales se llaman \textbf{canales binarios simétricos}, y si llamamos $p$ a la probabilidad de error en el canal, si $x=x_1\cdots x_n\in C$ es una palabra clave de un código, la probabilidad de ser enviado sin error es
\[
	P(\pull{x} = \push{x}) =P(\pull{x_1} = \push{x_1})\cdots P(\pull{x_n} = \push{x_n})=(1-p)^n
\]

Pero la pregunta que realmente hay que preguntarse y que será el objeto principal de este trabajo es:

\textbf{¿Cual es la probabilidad de que C sea capaz de corregir una palabra recibida?}

\subsection{P}

Un buen $(n, m, d)$-código debe tener las siguientes características:
\begin{itemize}
	\item Un valor pequeño de $n$, pues así se transmitirá más rápido.
	\item Un valor alto de $m$, pues así se enviará más información.
	\item un valor alto de $d$, para corregir la mayor cantidad de errores.
\end{itemize}

Como todas estas características no son posibles, se tiene que valorar que valor se quiere optimizar.
¿Que relación hay entre estos valores?

En una primera estimación, se tiene que $1\leq m\leq 2^n$ y $1\leq d\leq n$.

%TODO: Desarrollar.