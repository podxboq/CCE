\DeclareMathOperator{\tr}{Tr}
\DeclareMathOperator{\diag}{Diag}
\newcommand{\matr}[1]{\mathbf{#1}}
\providecommand{\tq}{\mid}
\providecommand{\N}{\mathbb{N}}
\providecommand{\Z}{\mathbb{Z}}
\providecommand{\Q}{\mathbb{Q}}
\providecommand{\R}{\mathbb{R}}
\providecommand{\C}{\mathbb{C}}
\providecommand{\H}{\mathcal{H}}
\providecommand{\Im}[1]{Im(#1)}
\providecommand{\Re}[1]{Re(#1)}
\providecommand{\conjugate}[1]{\bar{#1}}
\providecommand{\pescalar}[2]{\langle #1,#2 \rangle}
\providecommand{\braket}[2]{\left\langle#1\mid#2\right\rangle}
\providecommand{\bra}[1]{\left\langle#1\right\rvert}
\providecommand{\ket}[1]{\left\lvert#1\right\rangle}
\providecommand{\ketbra}[2]{\left\lvert#1\right\rangle\!\left\langle#2\right\rvert}
\providecommand{\so}{\Rightarrow}
\providecommand{\by}[1]{\overset{\fbox{\tiny #1}}{=}}
\providecommand{\byref}[1]{\overset{\fbox{\tiny\ref{#1}}}{=}}
\providecommand{\maps}[3]{#1:#2\longrightarrow #3}
\providecommand{\coma}{,\thinspace}
\providecommand{\pari}[2]{(#1,\thinspace #2)}
\providecommand{\indexdots}[3]{#1=#2,\ldots,#3}
\providecommand{\define}[2]{\textbf{#1}\label{def:#2}}
\providecommand{\avg}[1]{\left\langle#1\right\rangle}
\providecommand{\abs}[1]{\lvert#1\rvert}
\providecommand{\nor}[1]{\lVert#1\rVert}
\providecommand{\operatoravg}[3]{\left\langle#1|#2|#3\right\rangle}
\providecommand{\vec}[1]{\overrightarrow{#1}}
\newcommand{\set}[1]{\left\{#1\right\}}
\newcommand{\where}{\mathrel{}\middle|\mathrel{}}
\newcommand{\ndots}[3]{#1 = #2, \dots, #3}
%Kets notables
\newcommand{\ketMas}{\frac{1}{\sqrt{2}}(\ket{0}+\ket{1})}
\newcommand{\ketMenos}{\frac{1}{\sqrt{2}}(\ket{0}-\ket{1})}
\newcommand{\ketIMas}{\frac{1}{\sqrt{2}}(\ket{0}+i\ket{1})}
\newcommand{\ketIMenos}{\frac{1}{\sqrt{2}}(\ket{0}-i\ket{1})}
\newcommand{\ketBellUno}{\frac{1}{\sqrt{2}}(\ket{00}+\ket{11})}
\newcommand{\ketBellDos}{\frac{1}{\sqrt{2}}(\ket{00}-\ket{11})}
\newcommand{\ketBellTres}{\frac{1}{\sqrt{2}}(\ket{10}+\ket{01})}
\newcommand{\ketBellCuatro}{\frac{1}{\sqrt{2}}(\ket{10}-\ket{01})}

%OPERADORES 2x2
\newcommand{\matrixX}{\begin{pmatrix}
	                      0 & 1 \\ 1 & 0
\end{pmatrix}}
\newcommand{\matrixY}{\begin{pmatrix}
	                      0 & -i \\ i & 0
\end{pmatrix}}
\newcommand{\matrixZ}{\begin{pmatrix}
	                      1 & 0 \\ 0 & -1
\end{pmatrix}}
\newcommand{\matrixH}{\frac{1}{\sqrt {2}}  \begin{pmatrix}
	                                           1 & 1 \\ 1 & -1
\end{pmatrix}}
%OPERADORES $4x4
\newcommand{\matrixCNOT}{\begin{pmatrix}
	                         1 & 0 & 0 & 0 \\ 0 & 0 & 0 & 1 \\ 0 & 0 & 1 & 0 \\ 0 & 1 & 0 & 0
\end{pmatrix}}
\providecommand{\logical}[2]{#1_{\shortrightarrow #2}}
\providecommand{\logicalGeneric}[1]{\logical{#1}{n}}
\providecommand{\palabra}[2]{#1_1\cdots #1_{#2}}
\providecommand{\palabraN}[1]{\palabra{#1}{n}}
\providecommand{\palabraG}{\palabra{x}{n}}
\providecommand{\push}[1]{\stackrel{\hookrightarrow}{#1}}
\providecommand{\pull}[1]{\stackrel{\hookleftarrow}{#1}}
\providecommand{\lowerInt}[1]{\lfloor #1 \rfloor}
\providecommand{\upperInt}[1]{\lceil #1 \rceil}