\section{Códigos perfectos}

Sobre $\Z_2^n$ se define las dos siguientes operaciones: $\forall x=\palabraG, y=\palabraN{y}\in\Z_2^n$
\begin{align*}
	x\mas y &= x_1+y_1\dots x_n+y_n\\
	x\por y &= x_1 y_1\dots x_n y_n
\end{align*}

\begin{definition}
	Sea $x\in\Z_2^n$, se define el \textbf{peso} de $x$ a $w(x)=\sum_{i=1}^n x_i$.
\end{definition}
\begin{lemma}
	Sean $x, y\in\Z_2^n$, se tienen los siguientes resultados:
	\begin{enumerate}
		\item $w(x) = d(0,x)$.
		\item $d(x,y)=w(x\mas y)$.
		\item $w(x\mas y)=w(x)+w(y)-2w(x\por y)$
	\end{enumerate}
\end{lemma}

\begin{lemma}
	Sean $n, M, d\in\N$ con $d$ impar.\ Existe $C$ un $(n, M, d)$-código binario si y sólo si existe $C'$ un $(n+1, M, d+1)$-código binario.
\end{lemma}
\begin{proof}
	$\so$) Dado $C=\set{x_1, \dots, x_M}$ un $(n, M, d)$-código binario, donde $x_i=x_{i1}\cdots x_{in}\in C$ podemos construir $C'=\set{y_1,\dots, y_M}$ un $(n+1, M, d+1)$-código binario definiendo

	\[
		y_i=\begin{cases}
			    x_{i1}\cdots x_{in}0 & \text{ si $w(x_i)$ es par}\\
			    x_{i1}\dots x_{in}1 & \text{ si $w(x_i)$ es impar}
		\end{cases}
	\]
	Por construcción se tiene que $w(y_i)$ siempre es par y por tanto se tendrá también que $d(y_i, y_j)$ es par pues
	\[
		d(y_i, y_j) = w(y_i) + w(y_j) - 2 w(y_i\por y_j)
	\]
	Por otra parte se tiene que
	\begin{align*}
		d(y_i, y_j) &= w(y_i\mas y_j)=\sum_{k=1}^{n+1} (y_{ik}\mas y_{jk})=\sum_{k=1}^{n} (y_{ik}\mas y_{jk})+(y_{in+1}\mas y_{jn+1})=\\
		&= \sum_{k=1}^{n} (x_{ik}\mas x_{jk})+(y_{in+1}\mas y_{jn+1})=d(x_i, x_j)+(y_{in+1}\mas y_{jn+1})\so\\
		&\so d(x_i, x_j)\leq d(y_i, y_j)\leq d(x_i, x_j)+1\\
		d(y_l, y_k) &= w(y_l\mas y_k)=\sum_{m=1}^{n+1} (y_{lm}\mas y_{M+1m})=\sum_{k=1}^{n} (x_{lm}\mas x_{km})+(0\mas 1)=\\
		&=w(x_l\mas x_k)+1=d(x_l, x_k)+1 = d + 1
	\end{align*}
	Como $d=r(C)$ existen al menos dos palabras clave $x_l, x_k$ tal que $d(x_l, x_k) = d$ y como por hipótesis este es impar entonces $w(x_l)$ y $w(x_k)$ tiene que ser uno par y el otro impar y por tanto $d(y_l, y_k) = d+1$.

	$\Leftarrow$) Para el recíproco sea $C$ un $(n+1, M, d+1)$-código binario y existe al menos dos palabras clave $x_l$ y $x_k$ tales que $d(x_l, x_k) = d+1$, llamemos $i$ al índice donde $x_{li}\neq x_{ki}$, y por trasposición llevemos $C$ a otro código equivalente, y por tanto con el mismo radio interior, para que $i=n+1$.

	Se construye $C'=\set{y_1,\dots y_M}$ de la siguiente forma
	\[
		y_i = x_{i1}\dots x_{in}
	\]
	Con el mismo desarrollo que arriba se tiene que $d(x_i, x_j)-1\leq d(y_i, y_j)\leq d(x_i, x_j)$ y así sólo falta comprobar que $d(y_l, y_k) = d(x_l, x_k)-1=d+1-1=d$.
\end{proof}

\begin{theorem}[Sphere-packing o Hamming bound]
	Un $(q, n, M, 2t+1)$-código satisface $M|B(0, t)|\leq 2^n$.
\end{theorem}

\begin{definition}
	Diremos que un $(q, n, M, 2t+1)$-código es un \define{código perfecto}{codigo_perfecto} si $M|B(0, t)|= 2^n$.
\end{definition}

Para buscar códigos perfectos no triviales, vamos a centrarnos en una familia de códigos llamados de bloques con diseños balanceados.

\subsection{Códigos de bloques con diseños balanceados}
\begin{definition}
Un \define{bloque con diseño balanceado}{bloque-diseño-balanceado} es un conjunto $S$ de $v$ elementos, llamados puntos, una colección de $b$ subconjuntos de $S$ llamados bloques y 3 valores $k, r$ y $\lambda$ tal que
	\begin{enumerate}
		\item Cada bloque contiene exactamente $k$ puntos.
		\item Cada punto pertenece a $r$ bloques.
		\item Cada par de puntos están juntos en $r$ bloques.
	\end{enumerate}
	Un código de estas características de denota un $(b, v, r, k, \lambda)$-diseño.
\end{definition}

\begin{example}
	Toma $S=\set{1, 2, 3, 4, 5, 6, 7}$ y los siguientes subconjuntos $S_1=\set{1, 2, 4}, S_2=\set{2, 3, 5}, S_3=\set{3, 4, 6}, S_4=\set{4, 5, 7}, S_5=\set{5, 6, 1}, S_6=\set{6, 7, 2}$ y $S_7=\set{7, 1, 3}$.
	\[
		\begin{tabular}{c|ccccccc}
			& 1 & 2 & 3 & 4 & 5 & 6 & 7\\
			\hline
			1& 3 & 1 & 1 & 1 & 1 & 1 & 1\\
			2& 1 & 3 & 1 & 1 & 1 & 1 & 1\\
			3& 1 & 1 & 3 & 1 & 1 & 1 & 1\\
			4& 1 & 1 & 1 & 3 & 1 & 1 & 1\\
			5& 1 & 1 & 1 & 1 & 3 & 1 & 1\\
			6& 1 & 1 & 1 & 1 & 1 & 3 & 1\\
			7& 1 & 1 & 1 & 1 & 1 & 1 & 3
		\end{tabular}
		\qquad
		\begin{tabular}{c|ccccccc}
			& $S_1$ & $S_2$ & $S_3$ & $S_4$ & $S_5$ & $S_6$ & $S_7$\\
			\hline
			1& 1 & 0 & 0 & 0 & 1 & 0 & 1\\
			2& 1 & 1 & 0 & 0 & 0 & 1 & 0\\
			3& 0 & 1 & 1 & 0 & 0 & 0 & 1\\
			4& 1 & 0 & 1 & 1 & 0 & 0 & 0\\
			5& 0 & 1 & 0 & 1 & 1 & 0 & 0\\
			6& 0 & 0 & 1 & 0 & 1 & 1 & 0\\
			7& 0 & 0 & 0 & 1 & 0 & 1 & 1
		\end{tabular}
	\]
	La primera tabla muestra en cuantos subconjuntos está incluido el conjunto $\set{i,j}$ y la segunda tabla muestra a qué conjunto pertenece $i$, visto como matrices tendríamos dos matrices $A=(a_{ij})$ y $B=(b_{ij})$ definidas por
	\begin{align*}
		a_{ij} & =|\set{k\tq \set{i,j}\subseteq S_k}|\\
		b_{ij} & =|\set{i}\cap S_j|
	\end{align*}
	La matriz $A$ nos permite comprobar los puntos 2 y 3 de la definición anterior y la matriz $B$ nos permite comprobar los puntos 1 y 2, con lo cual podemos ver que tenemos un $(7, 7, 3, 3, 1)$-diseño.
\end{example}

\begin{definition}
	Sea un $(b, v, r, k, \lambda)$-diseño, con bloques $B_1,\dots, B_v$.
	Definimos la \define{matriz de incidencia}{matriz-incidencia} de $C$ a la matriz $A=(a_{ij})\in M_{v\times b}(\Z_2)$ definida por $a_{ij} = |\set{i}\cap B_j|$.
\end{definition}

\begin{example}
	Volviendo al ejemplo anterior, consideremos el conjunto $C=\set{\logical{0}{7}, \logical{1}{7}}\cup \set{a_i}_{i=1}^7\cup \set{b_i}_{i=1}^7$ donde $a_i$ son las filas de la matriz incidencia $B$ y $b_i = -a_i$.
	Esta construcción nos permite tener un $(7, 16 ,3)$-código perfecto
\end{example}

\begin{lemma}
	Sea un $(b, v, r, k, \lambda)$-diseño.\ Se cumple:
	\begin{itemize}
		\item $bk=vr$.
		\item $r(k-1)=\lambda(v-1)$.
		\item $v\leq b$.
	\end{itemize}
\end{lemma}

\begin{definition}
	Un $(b, v, r, k, \lambda)$-diseño es un \define{diseño simétrico}{diseño-simetrico} si $b=v$.
	Por el resultado anterior, se tendrá también que $k=r$ y por tanto podemos denotar a estos diseños como un $(v, k, \lambda)$-diseño.
\end{definition}