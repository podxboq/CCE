\section{Códigos equivalentes}

Un buen $(q, n, m, d)$-código debe tener las siguientes características:
\begin{itemize}
	\item Un valor pequeño de $n$, pues así se transmitirá más rápido.
	\item Un valor alto de $m$, pues así se enviará más información.
	\item Un valor alto de $d$, para corregir la mayor cantidad de errores.
\end{itemize}

Como todas estas características no son posibles, se tiene que valorar que valor se quiere optimizar.
¿Que relación hay entre estos valores?

En una primera estimación, se tiene que $1\leq m\leq q^n$ y $1\leq d\leq n$.
Para empezar, vamos a ver como se puede optimizar $m$ cuando fijamos el resto de valores.

\begin{definition}
	Se denota por $M = M_q(n, d)$ al mayor valor para el cual existe un $(q, n, M, d)$-código.
\end{definition}

Es fácil comprobar que $M_q(n, 1)=q^n$, pues es el número máximo de palabras.
Y por otro lado $M_q(n, n)=q$ ya que por ejemplo, en la posición 1 no puede haber $q+1$ elementos distintos, teniendo así una cota superior, pero como ya se ha visto en el capítulo anterior, tenemos los $q$-códigos de repetición de longitud $n$, lo que nos da la igualdad.

Para estudiar la existencia de códigos y sus propiedades, no tenemos que conocer todos los códigos existentes, sino caracterizar sus propiedades y estudiar el representante de cada tipo de código, para ello vamos a definir el concepto de equivalencia entre código.

Recordemos que dado un alfabeto $\Gamma$, se denota por $S_\Gamma$ al grupo de las permutaciones de $\Gamma$ y que cuando $\Gamma = \Z_q$ entonces se denota por $S_q$.

\subsection{Permutación de palabras}

\begin{definition}
	Sea $\Gamma$ un alfabeto y $\sigma\in S_n$, se define la \textbf{permutación} de palabras como $\maps{P_{\sigma}}{\Gamma^n}{\Gamma^n}$ a la aplicación definida por
	\[
		P_{\sigma}(\palabraG)=x_{\sigma(1)}\cdots x_{\sigma(n)}
	\]
\end{definition}

Este tipo de aplicación, realiza un cambio en las posiciones de los elementos de la palabra.

\begin{lemma}
	\label{res:permutacion-igualdad-distancia}
	Sea $\Gamma$ un alfabeto y $\sigma\in S_n$, $\forall x,y\in\Gamma^n\so d(x, y) = d(P_\sigma(x), P_\sigma(y))$.
\end{lemma}
\begin{proof}
	Recordemos que $d(x, y) = n-\sum_{i=1}^n \delta_{x_i y_i}$ y que como $\sigma$ es biyectiva se tiene que $\existsUnique j\in\Z_n\tq j = \sigma^{-1}(i)\so x_i=x_{\sigma(j)}\so \delta_{x_i y_i}=\delta_{x_{\sigma(j)} y_{\sigma(j)}}$ y por tanto
	\[
		d(x, y) = n-\sum_{i=1}^n \delta_{x_i y_i}=n-\sum_{j=1}^n \delta_{x_{\sigma(j)} y_{\sigma(j)}}=d(P_\sigma(x), P_\sigma(y))
	\]
\end{proof}

\begin{lemma}
	Sea $\Gamma$ un alfabeto y $\sigma\in S_n$, $\forall x\in\Gamma^n\so d(x, P_\sigma(x))\leq\epsilon(\sigma)$.
\end{lemma}
\begin{proof}
	Sea $A\subset\Z_n$ el conjunto de los elementos que quedan invariantes por $\sigma$ y $B$ el conjunto complementario, por lo tanto, $n = |A|+|B|$, se tiene
	\begin{itemize}
		\item $\forall i\in A\ i = \sigma(i)\so x_i =x_{\sigma(i)}\so \delta_{x_i x_{\sigma(i)}}=1\so \sum_{i\in A} \delta_{x_i x_{\sigma(i)}}=|A|$.
		\item Si $i\in B\ i \neq \sigma(i)$ pero no podemos garantizar que $x_i \neq x_{\sigma(i)}$ así que sólo podemos decir que se puede llamar $x = \sum_{i\in B} \delta_{x_i x_{\sigma(i)}}\geq 0$.
	\end{itemize}
	Como $|B|=\epsilon(\sigma)$ se tiene que
	\[
		d(x, P_\sigma(x)) = n-\sum_{i=1}^n \delta_{x_i x_{\sigma(i)}}=n-\sum_{i\in A} \delta_{x_i x_{\sigma(i)}}-\sum_{i\in B} \delta_{x_i x_{\sigma(i)}}=n-|A|-x=|B|-x\leq |B|=\epsilon(\sigma)
	\]
\end{proof}

\subsection{Transmutación de palabras}

\begin{definition}
	Sea $\Gamma$ un alfabeto y $\sigma\in S_\Gamma$, se define la \textbf{transmutación sobre el índice $i$} de palabras como $\maps{T_{\sigma_i}}{\Gamma^n}{\Gamma^n}$ a la aplicación definida por
	\[
		T_{\sigma_i}(\palabraG)=y_1\cdots y_n \text{ donde } y_k = \begin{cases}
			                                                           \sigma(x_i) \text{ si } k = i \\
			                                                           x_k \text{ resto}
		\end{cases}
	\]
\end{definition}

Al igual que para las permutaciones de palabra, se tienen las siguientes propiedades

\begin{lemma}
	\label{res:transmutacion-igualdad-distancia}
	Sea $\Gamma$ un alfabeto y $\sigma\in S_\Gamma$, $\forall i\leq n, \forall x,y\in\Gamma^n\so d(x, y) = d(T_{\sigma_i}(x), T_{\sigma_i}(y))$.
\end{lemma}
\begin{proof}
	Recordemos que $d(x, y) = n-\sum_{i=1}^n \delta_{x_i y_i}$ y que como $\sigma$ es biyectiva se tiene que $x_i=y_i\iff \sigma(x_i) = \sigma(y_i)\so \delta_{x_i y_i}=\delta_{\sigma(x_i) \sigma(y_i)}$ y por tanto
	\[
		d(x, y) = n-\sum_{k=1}^n \delta_{x_k y_k}=n-\sum_{\substack{k=1\\k\neq i}}^n \delta_{x_k y_k}-\delta_{x_i y_i}=n-\sum_{\substack{k=1\\k\neq i}}^n \delta_{x_k y_k}-\delta_{\sigma(x_i) \sigma(y_i)}=d(T_{\sigma_i}(x), T_{\sigma_i}(y))
	\]
\end{proof}

\begin{lemma}
	Sea $\Gamma$ un $q$-alfabeto y $\sigma\in S_\Gamma$, $\forall i\leq n\ \forall x\in\Gamma^n\so d(x, T_{\sigma_i}(x))\leq 1$.
\end{lemma}
\begin{proof}
	\[
		d(x, T_{\sigma_i}(x)) = n-\sum_{\substack{k=1\\k\neq i}}^n\delta_{x_k x_k}- \delta_{x_i \sigma(x_i)}=n-(n-1)-\delta_{x_i \sigma(x_i)}\leq 1
	\]
\end{proof}

Para hacer referencia a alguna de las dos aplicaciones definidas anteriormente, se refiere a ambas como \textbf{transformaciones de palabras}, y al conjunto de ellas por $PT(\Gamma)$.

Para terminar esta sección veamos como transformar cualquier palabra en otra dada.

\begin{lemma}
	\label{res:palabra-transforma-cero}
	Sea $\Gamma$ un alfabeto y $x,y\in\Gamma^n$.
	Existen $\sigma_1,\cdots,\sigma_n\in PT(\Gamma)$ tal que $T_{{\sigma_n}_n}\circ\cdots\circ T_{{\sigma_1}_1}(x)=y$.
\end{lemma}
\begin{proof}
	Sean $x=\palabraG$ e  $y=\palabraN{y}$ dos elementos de $\Gamma^n$.
	Se define $\maps{\sigma_k}{\Gamma}{\Gamma}$ por
	\[\sigma_k(x)=\begin{cases}
		              y_k \text{ si } x = x_k\\
		              x_k \text{ si } x = y_k\\
		              x \text{ resto }
	\end{cases}
	\]

	Así aplicando secuencialmente
	\begin{align*}
		T_{{\sigma_1}_1}(x_1 x_2\cdots x_n)&=\sigma_1(x_1)x_2\cdots x_n=y_1 x_2\cdots x_n\so\\
		T_{{\sigma_2}_2}(y_1 x_2\cdots x_n)&=y_1\sigma_2(x_2)\cdots x_n=y_1 y_2\cdots x_n\so\\
		\vdots & \\
		T_{{\sigma_n}_n}(y_1 y_2\cdots x_n)&=y_1 y_2\cdots\sigma_n(x_n)=y_1 y_2\cdots y_n
	\end{align*}
\end{proof}

\subsection{Códigos equivalentes}
Cuando se tiene un código $C$, y $\sigma$ una transformación de palabra, aplicar la transformación a todos los elementos del código, se obtiene el conjunto, $\sigma(C)=\set{\sigma(x) \forall x\in C}$.
Ya se tiene todos los elementos necesarios para definir la equivalencia de códigos.
\begin{definition}
	Sea $\Gamma$ un alfabeto, y $C,C'\subset \Gamma^n$ dos códigos, diremos que $C$ está en relación con $C'$ si existen $\sigma_1,\cdots,\sigma_m$ transformaciones de palabras tal que $\sigma_m\circ\cdots\circ \sigma_1(C)= C'$.
\end{definition}

Veamos que la propiedad de estar en relación es una relación de equivalencia.
\begin{enumerate}
	\item \textbf{Reflexiva}.\ Ya que la identidad $I$ es una permutación de $\Z_n$, la permutación $P_I$ es una transformación de palabra y es el elemento identidad, por lo tanto $C$ esta en relación consigo mismo.
	\item \textbf{Simétrica}.\ Todas la permutaciones $\sigma$ de $\Z_n$ y $\theta$ de $\Gamma$ son invertibles cumpliendo que $P_\sigma^{-1}=P_{\sigma^{-1}}$ y $T_{\theta_i}^{-1}=T_{\theta_i^{-1}}$ se tiene que todas las transformaciones de palabras son invertibles, si $\sigma_m\circ\cdots\circ \sigma_1(C)= C'$ entonces $\sigma_1^{-1}\circ\cdots\circ \sigma_m^{-1}(C')= C$.
	\item \textbf{Transitiva}.\ Se tiene 3 código $C, C', C''$ tales que $\sigma_m\circ\cdots\circ \sigma_1(C)= C'$ y $\theta_l\circ\cdots\circ \theta_1(C')= C''$ entonces $\theta_m\circ\cdots\circ \theta_1\circ\sigma_m\circ\cdots\circ \sigma_1(C)= C''$
\end{enumerate}

Ahora ya se puede decir que cuando existe una combinación de transformaciones de palabras que lleve un código a otro, que ambos códigos son equivalentes.

Como consecuencia de los resultados~\eqref{res:permutacion-igualdad-distancia} y~\eqref{res:transmutacion-igualdad-distancia} se tiene que
\begin{lemma}
	Si $C$ y $C'$ son dos códigos equivalentes entonces $r(C)=r(C')$.
\end{lemma}
Al aplicar el resultado~\eqref{res:palabra-transforma-cero} se tiene que
\begin{lemma}
	Todo código es equivalente a un código que contenga una palabra fija.
\end{lemma}

\begin{example}
	El código binario $C=\set{00100, 00011, 11111, 11000}$ se puede transformar en el código equivalente  $C'=\set{00000, 01101, 10110, 11011}$.

	Las transformaciones de palabras que permiten llegar a $C'$ no son únicas, en este caso se puede hace con dos transformaciones, por ejemplo con $\maps{\sigma}{\N_5}{\N_5}$ definido por
	\[
		\sigma(x)=\begin{cases}
			          4 \text{ si } x=2\\
			          2 \text{ si } x=4\\
			          x \text{ resto}
		\end{cases}
	\]
	Y la transformación binaria $\maps{\theta}{\set{0, 1}}{\set{0, 1}}$ definida por $\theta(x)=\neg x$.
	\begin{center}
		\begin{tabular}{ccc}
			C     & $T_{\theta_3}(C)$ & $P_\sigma(T_{\theta_3}(C))$ \\
			00100 & 00000             & 00000                       \\
			00011 & 00111             & 01101                       \\
			11111 & 11011             & 11011                       \\
			11000 & 11100             & 10110
		\end{tabular}
	\end{center}
\end{example}